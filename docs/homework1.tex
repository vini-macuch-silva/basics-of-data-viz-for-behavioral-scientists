% Options for packages loaded elsewhere
\PassOptionsToPackage{unicode}{hyperref}
\PassOptionsToPackage{hyphens}{url}
%
\documentclass[
  12pt,
]{article}
\usepackage{lmodern}
\usepackage{amssymb,amsmath}
\usepackage{ifxetex,ifluatex}
\ifnum 0\ifxetex 1\fi\ifluatex 1\fi=0 % if pdftex
  \usepackage[T1]{fontenc}
  \usepackage[utf8]{inputenc}
  \usepackage{textcomp} % provide euro and other symbols
\else % if luatex or xetex
  \usepackage{unicode-math}
  \defaultfontfeatures{Scale=MatchLowercase}
  \defaultfontfeatures[\rmfamily]{Ligatures=TeX,Scale=1}
\fi
% Use upquote if available, for straight quotes in verbatim environments
\IfFileExists{upquote.sty}{\usepackage{upquote}}{}
\IfFileExists{microtype.sty}{% use microtype if available
  \usepackage[]{microtype}
  \UseMicrotypeSet[protrusion]{basicmath} % disable protrusion for tt fonts
}{}
\makeatletter
\@ifundefined{KOMAClassName}{% if non-KOMA class
  \IfFileExists{parskip.sty}{%
    \usepackage{parskip}
  }{% else
    \setlength{\parindent}{0pt}
    \setlength{\parskip}{6pt plus 2pt minus 1pt}}
}{% if KOMA class
  \KOMAoptions{parskip=half}}
\makeatother
\usepackage{xcolor}
\IfFileExists{xurl.sty}{\usepackage{xurl}}{} % add URL line breaks if available
\IfFileExists{bookmark.sty}{\usepackage{bookmark}}{\usepackage{hyperref}}
\hypersetup{
  pdftitle={Homework 1},
  hidelinks,
  pdfcreator={LaTeX via pandoc}}
\urlstyle{same} % disable monospaced font for URLs
\usepackage[margin=1in]{geometry}
\usepackage{graphicx,grffile}
\makeatletter
\def\maxwidth{\ifdim\Gin@nat@width>\linewidth\linewidth\else\Gin@nat@width\fi}
\def\maxheight{\ifdim\Gin@nat@height>\textheight\textheight\else\Gin@nat@height\fi}
\makeatother
% Scale images if necessary, so that they will not overflow the page
% margins by default, and it is still possible to overwrite the defaults
% using explicit options in \includegraphics[width, height, ...]{}
\setkeys{Gin}{width=\maxwidth,height=\maxheight,keepaspectratio}
% Set default figure placement to htbp
\makeatletter
\def\fps@figure{htbp}
\makeatother
\setlength{\emergencystretch}{3em} % prevent overfull lines
\providecommand{\tightlist}{%
  \setlength{\itemsep}{0pt}\setlength{\parskip}{0pt}}
\setcounter{secnumdepth}{-\maxdimen} % remove section numbering
\usepackage{setspace}\onehalfspacing

\title{Homework 1}
\author{}
\date{\vspace{-2.5em}}

\begin{document}
\maketitle

\hypertarget{homework-1}{%
\section{Homework 1}\label{homework-1}}

For this assignment, your task is to explore and then plot the
familiarity ratings, which is one of the measures from the study we
looked at in class. Much like we did in class for the transparency
measures, you are to explore the data, looking at the main contrast of
interest between metaphors and metonomies as well as at other
potentially interesting comparisons. You should produce two sets of
graphs:

\begin{itemize}
\item
  \textbf{Set 1} consists of exploration graphs, where you should plot
  the data using two different geoms and at least one grouping variable,
  to be chosen between \textbf{state} and \textbf{educ}. In total, you
  should create four plots, two focused on the main contrast of
  interest, each with a different geom, and two including at least one
  further variable;
\item
  \textbf{Set 2} consists of camera-ready graphs, where you should pick
  one pattern/ contrast of your choice and plot it using two different
  geoms. Importantly, your task is to make the graphs look as bad as
  possible while the data itself is plotted correctly. For that, you
  will have to explore different combinations of visualization
  techniques. Make extensive use of \texttt{fill}, \texttt{color},
  facetting, and remember to also modify the legend as well as the plot
  axes (think of axes labels, ticks, etc.).
\end{itemize}

Use \texttt{ggsave()} to export your final plots as .png files. Upload
your files to Stud.IP as a single .zip file containing two folders, one
for each set of graphs. The .zip file should be named \textgreater{}
homework1\_InitialSurname.

\end{document}
